\chapter{Conclusion and Future Work}

This thesis has presented GERT as part of an effort to investigate the efficacy of using
a type-safe, garbage-collected, high-level language on a multi-core embedded system. The premise
of this idea is that OS kernels provide redundant and costly isolation to programs which
do not require it because they are written in a HLL. GERT demonstrates, through micro benchmarks,
that removing the OS
from an embedded platform and running the high-level code directly on bare metal, can be
more performant than even a user space C program in Linux. Writing embedded programs with GERT
is also relatively painless because of Go's concurrency patterns and its portable standard library.
The goal of this thesis is not to convince everyone to use GERT, but rather to show that performant
embedded programs do not have to be written in C anymore. HLLs can provide memory safety and concurrency abstractions
even while running on bare metal.

The GERT project still leaves a few questions unanswered. First of all, GERT was
never bench marked against bare metal C running on the iMX6. This is because there was not enough time to develop an
RTOS that could provide similar semantics to Go. Instead, a Teensy with a cortex M4 processor was used
to establish an ideal latency baseline for an embedded system. GERT is not likely to outperform a well-written
RTOS, but GERT programs will still be more portable.

It is also possible that GERT can exist as
a kernel module inside of Linux, instead of a stand-alone bootable binary. Since GERT is already
memory-safe, it should be able to run in kernel mode without crashing the whole system. This is
an especially desirable avenue to approach because it means that an embedded programmer can have
the convenience of developing a GERT program in user space before installing it as a kernel
module where it can run more efficiently. Re-spinning GERT as a kernel module may also
allow it to hook into existing Linux drivers for the network card and the file system, greatly reducing
the number of drivers that have to be written for each platform.
