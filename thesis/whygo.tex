\chapter{Introduction}

Modern embedded systems are composed of multicore SOCs that require
careful thought about race conditions and event serialization. C is the
most commonly used language to program such low level systems, but it is
a double-edged blade. Low level code written in C can more easily and
directly interface with hardware, but it can also be plagued with
difficult bugs, especially concurrency-related bugs. Concurrent
user-space programs, on the other hand, are usually written in a
high-level language, such as Rust or Go, which abstracts concurrency
and provides memory safety.

Embedded programmers are often drawn towards using Linux for their work
because then they can write their embedded program in user space with a
high level language. Popular platforms that use this paradigm include
the Raspi and Beaglebone. However, embedded programs that run in user space suffer
from significant event latency because external interrupts
must shuffle their way through the kernel and the kernel's scheduler
will also constantly preempt the program. Usually, an experienced programmer
does not even want or need the kernel, but they pay the price just to run the
high-level code.

There are ongoing efforts to bring high-level languages to desktop
operating system kernels and also single-core microcontrollers, but
there is no known system which provides a high-level language environment for
multicore SOCs. Singularity <cite> and Biscuit <cite> are desktop
operating system kernels, written in Sing\# and Go, which focus on
hosting user-space programs. Copper <cite> and MicroPython <cite>
are small embedded toolkits, written in Rust and Python, which aim
to provide a high-level programming environment for single-core
microcontrollers. Multicore SOCs have, so far, been left out of the
picture. This thesis presents a new embedded toolkit, G.E.R.T, which
is specifically intended for concurrent, bare-metal embedded applications.

G.E.R.T is a bare-metal, Go-based embedded toolkit for multicore ARMv7a processors.
It was developed in order to make bare-metal programming
easier for ARMv7a SOCs with the help of Go's channels and goroutines.
G.E.R.T can run on a single-core processor but its effectiveness is substantially
reduced because any blocking operation can lock the whole system. On the other hand,
G.E.R.T will automatically scale to utilize all available cpus in multicore systems
because the Go runtime automatically scales to all available cpus. It is unlikely
that a low-level system implemented with G.E.R.T will outperform the same system written in
good C, but results show that G.E.R.T does outperform userspace
C code in Linux.

One particular concern about using a garbage-collected language
for low-level code is GC pause times. This is not an issue in G.E.R.T because interrupt
handlers can execute even when the world is stopped as long as the interrupt code
does not call the Go scheduler. This is an acceptable constraint for embedded
environments because any interrupt code that calls the Go scheduler is performing
an operation that could potentially block, and blocking operations should not
be happening during interrupt time in the first place.


\section{Why Write Low Level System Code in Go?}

At first glance, Go code looks a lot like C. There are no
classes and every object has a type which is known at compile time.
This already makes Go a good systems language, but Go's
greatest feature is its built-in support for concurrency through
goroutines and channels. Goroutines are lightweight threads that
the Go runtime can schedule without help from the operating system.
Channels are typed FIFO queues which can help to serialize
asynchronous events, perhaps coming from several goroutines.
With these features, Go is like an updated version
of C for multicore systems, but without buffer overflows and
null pointer dereferences.

Go's implementation mirrors that of a small
real-time OS. Go's threads are lightweight and
cooperatively scheduled so that execution only transfers during
blocking operations. The runtime also manages its own pool of memory
and exports its own atomic primitives through the standard "sync"
package. In fact, Go provides most of the common OS primitives natively
in its standard libraries <https://golang.org/pkg/\#stdlib>. This means
that G.E.R.T can also provide most of the convenience of a full-blown kernel
without latency degredation.

Embedded systems are increasingly relying on dedicated peripherals to
provide service, instead of very fast cpus. SOCs contain dedicated
silicon peripherals to help with everything from serial communication to
interrupt priority filtering. These peripherals free the cpus from bit-banging
high-frequency signals so they can spend more time directing program
flow instead. Go fits in well with such a system because its goroutines can be
used to concurrently monitor state and channels can be used to relay that information
back to a central coordinator. When an output must be switched, G.E.R.T simply
issues a driver call that changes the behavior of a peripheral.

\section{Outline}
Outline the rest of the thesis

%Stay tuned!
%Modern embedded systems are composed of multicore SOCs that require
%careful thought about race conditions and event serialization. Like
%operating system kernels, most of these embedded systems are still implemented
%in C because it is a simplistic language that makes it good for "bare-metal"
%development.
%
%
%outline:
%computers are quicker and multicore programming is scary, but everyone still uses C
%
%C's simplicity makes it error prone for concurrent programs.
%Because C is very simple, the programmer must implement additional complexities in order to write
%concurrent programs.
%
%try2:
%Low-level system code has been written in C since the 1970's because it is powerful
%and reliable. C can be used to express any operation a computer can do and it can also be
%compiled to fast byte code. Once, during an interview, an engineer even remarked: "If you
%can't do it in C, you can't do it". This does not mean that C is always the best choice though.
%
%Even though multicore systems are commonplace now, kernels are still written in C.
%
%Writing complex concurrent programs in C is too hard. Because C is very simple,
%the programmer must implement additional complexities in order to write
%concurrent programs.
%
%
%There are very
%few built-in abstractions so it is left to the programmer to layer additional complexities
%in order to accomplish a task.
%
%
%try 1:
%Low-level system code has been written in C since the 1970's because C is powerful
%and reliable. C can be used to express any operation a computer can do and it doesn't come
%with any baggage like languages with a runtime do. C is also easy to learn because it doesn't
%require advanced degrees in order to comprehend, like Haskell and Coq do. The problems with C
%only begin to show when concurrency comes into play. C, by itself, has no idea of concurrency
%or concurrent programming patterns. It is really all up to the programmer to lay down these
%abstractions. Combined with the burden of manual memory management, concurrent programming in C
%almost always results in pouring over JTAG trace logs for hints of a race condition.
%  Faced with this bleak outlook, perhaps it is reasonable to take a performance hit in exchange for
%faster development and less bugs. After all, computers have gotten significantly quicker in
%the last 20 years. This is where Go can come in. Go is meant to be a systems language
%that provides fundamental support for concurrency and cummincation
