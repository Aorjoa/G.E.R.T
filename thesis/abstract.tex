% $Log: abstract.tex,v $
% Revision 1.1  93/05/14  14:56:25  starflt
% Initial revision
% 
% Revision 1.1  90/05/04  10:41:01  lwvanels
% Initial revision
% 
%
%% The text of your abstract and nothing else (other than comments) goes here.
%% It will be single-spaced and the rest of the text that is supposed to go on
%% the abstract page will be generated by the abstractpage environment.  This
%% file should be \input (not \include 'd) from cover.tex.

Embedded systems are becoming increasingly complicated due
to the emergence of low-power SOC's with multiple cores and
complicated virtual memory systems that mirror that of x86,
but performant embedded programs are still largely written from scratch in
C. This leads to constant re-implementation of schedulers and virtual memory systems
which is a waste of time. RealTime Operating Systems (RTOS's) do exist
but they impose their own driver models and are written in C, increasing the likelyhood
of bugs.\par
  This thesis explores a modified a version of the Go runtime intended to run
bare-metal on an ARMv7a system-on-a-chip (SOC) in order to evaluate
the effectiveness of using a high-level, type-safe, and garbage collected
language for embedded applications. G.E.R.T, the Golang Embedded RunTime,
provides the multiprocessor support and basic memory abstractions of a
typical RTOS while also freeing the user to leverage the language features
of Go in order to develop custom driver models and build more complicated
concurrent programs that are easier to reason about than similar ones
written in C.

