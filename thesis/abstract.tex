% $Log: abstract.tex,v $
% Revision 1.1  93/05/14  14:56:25  starflt
% Initial revision
% 
% Revision 1.1  90/05/04  10:41:01  lwvanels
% Initial revision
% 
%
%% The text of your abstract and nothing else (other than comments) goes here.
%% It will be single-spaced and the rest of the text that is supposed to go on
%% the abstract page will be generated by the abstractpage environment.  This
%% file should be \input (not \include 'd) from cover.tex.
\section{abstract}

Embedded systems are becoming increasingly complicated due
to the emergence of SOC's (system-on-a-chip) with multiple cores, dizzying amounts of peripherals, and
complicated virtual memory systems. Despite this,
performant embedded programs are still largely written from scratch in
C, which leads to constant re-implementation of the same subsystems and difficult bugs.
\\


This thesis explores a new system called G.E.R.T, the Golang Embedded RunTime, for ARM processors.
GERT is a modified version of the Go runtime for
bare-metal operation on ARMv7a SOC's in order to evaluate
the effectiveness of using a high-level, type-safe, and garbage collected
language for embedded applications. G.E.R.T
provides the multiprocessor support and basic memory abstractions of a
typical embedded toolkit while also freeing the user to leverage the language features
of Go in order to develop
concurrent embedded programs that are easier to reason about than similar ones
written in C.

