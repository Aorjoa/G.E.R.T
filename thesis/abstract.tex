% $Log: abstract.tex,v $
% Revision 1.1  93/05/14  14:56:25  starflt
% Initial revision
% 
% Revision 1.1  90/05/04  10:41:01  lwvanels
% Initial revision
% 
%
%% The text of your abstract and nothing else (other than comments) goes here.
%% It will be single-spaced and the rest of the text that is supposed to go on
%% the abstract page will be generated by the abstractpage environment.  This
%% file should be \input (not \include 'd) from cover.tex.
%%\section{abstract}

Embedded systems are becoming increasingly complicated due to the emergence of
SOCs (system-on-a-chip) with multiple cores, dizzying amounts of peripherals, and
complicated virtual memory systems. Unfortunately, performant embedded systems
for SOCs are still largely written in either bare-metal C or userspace C because
high-level languages running in userspace can have too much latency.

This thesis proposes a new system called G.E.R.T, the Golang Embedded RunTime,
for multi-core ARM processors.
GERT is a modified version of the Go runtime for
bare-metal operation on multi-core ARMv7a SOC's. It is used to evaluate
the effectiveness of using a high-level, type-safe, and garbage collected
language for embedded applications. G.E.R.T
provides the multiprocessor support and basic memory abstractions of a
typical embedded toolkit while also enabling the user to leverage the language features
of Go in order to develop
concurrent embedded programs that are easier to reason about than similar ones
written in C.

