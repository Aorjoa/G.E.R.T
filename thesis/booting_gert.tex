\chapter{Booting the Go Runtime}

<insert figure from proposal that shows boot process>

Even though Go code is compiled, it relies on a runtime to coordinate certain actions with the OS.
Timers, locks, and file descriptors are just a few of the OS abstractions that the runtime hinges on
in order to function at all. This means that getting compiled Go code to run bare metal on an SOC requires
more than just a boot loader, the Go runtime itself must be modified to work without any OS abstractions.
This poses a bootstrapping problem because any modifications made to the Go runtime's initialization
process must not inadvertently cause it to use an abstraction that does not yet exist. For example,
creating a new object with $make()$ would be disasterous if the GC has not yet been initialized.
In observation of these constraints, G.E.R.T boots via a 3-stage process. The first stage is u-boot, which
configures the clocks and copies the second stage off an sdcard into memory before jumping into it. The second
stage bootloader is a small C program which contains the entire G.E.R.T kernel ELF in its data section. This stage sets
up the inital Go stack and loads the G.E.R.T ELF into memory before jumping to its entry point. The third stage
of the bootloader lives inside G.E.R.T and is mostly written in Go, along with some Plan 9 assembly. It finishes the
boot process.



Working off the
initial stack from stage 2, the stage 3 bootloader enumerates all of RAM into page tables and creates an idenity mapping
with a new stack before turning on the MMU. After this, a thread scheduler is setup and synchronization primitives, like
$futex()$ are enabled. Additional CPU's are booted in main after the Go runtime has finished initializing.

\section{System Specification}
G.E.R.T is written on a Freescale i.MX6 Quad SOC which implements the (32 bit) ARMv7a Cortex A9 MPCore architecture.
The SOC sits on a Wandboard Quad baseboard. The i.MX6 Quad has 2GB of RAM and a wealth of memory mapped peripherals.
Look at the memory map below. The rest of this chapter will discuss the implementation details of booting and
running the Go runtime bare-metal on this SOC.

<memory map of imx6 here>

<memory map of GERT here>

\section{Stage 1 Bring Up}
The u-boot loader is used to initialize device clocks and load the G.E.R.T bootloader into RAM.
When the SOC is powered on, the program counter starts executing from ROM. The code in the ROM reads
u-boot into RAM and jumps into it. Unlike desktop PCs and laptops which use a BIOS to configure the
frequency dividers and RAM timings, the iMX6 has no such thing so u-boot does it instead.
U-boot programs the myriad of frequency dividers which are required
to run the i.MX6 at a frequency of 792MHz per core. After this, u-boot loads the G.E.R.T
kernel off the sdcard and into RAM at address 0x50000000. This address is specifically chosen because it
does not overlap with any ELF program headers of the G.E.R.T kernel which are loaded in stage 2. After
the stage 2 bootloader is in RAM, uboot jumps into it.

\section{Stage 2 Kernel Decompression}
The G.E.R.T bootloader sets up the initial Go stack and decompresses the Go kernel
ELF into RAM. Much like Linux, the kernel of G.E.R.T is wrapped in a custom
bootloader stage. This is necessary because G.E.R.T is compiled as a user space
Linux program which expects a stack and the standard ELF auxiliary vectors. By
default, the Go compiler links all programs at address 0x0. This would normally
be a disaster for the i.MX6 because the first megabyte of RAM is either inaccessable
or reserved for peripherals. One solution around this is to simply turn on the MMU
in the stage 2 bootloader but this creates a headache with preserving page tables
across the transition to Go. An alternative, and much simpler, solution is to
just change the link address of the Go ELF. This is the preferred approach so
G.E.R.T's build system uses a link address of 0x10000000, which is the start of
RAM on the iMX6. After loading the Go binary into RAM, the stage 2 bootloader reserves
4kb of initial stack and jumps into G.E.R.T.

\section{Stage 3 Go Runtime}
The thread scheduler and virtual memory system are statically initialized
in order to prevent Go runtime subsystems from running before the environment
is ready. At the beginning of execution, G.E.R.T is in a constrained
situation: Linux is not there but the Go runtime thinks it is. Specifically, there
is no scheduler, no virtual memory, no syscalls. Nothing but a 4kb stack
and the program counter. This is clearly inadequate for the Go runtime
to do anything but crash, so G.E.R.T creates all of these missing subsystems
,in Go, before the runtime actually uses them.


\subsection{Virtual Memory Setup}
\subsection{Thread Scheduling and Trapframes}
\subsection{Interrupts}
\subsection{Keeping Time}
\subsection{Booting Secondary CPUs}

